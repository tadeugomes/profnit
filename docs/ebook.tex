% Options for packages loaded elsewhere
\PassOptionsToPackage{unicode}{hyperref}
\PassOptionsToPackage{hyphens}{url}
%
\documentclass[
]{book}
\title{Pesquisa Tecnológica: Qualitativa e Quantitativa \& Estatística Aplicada a PI\&TT}
\author{Equipe de Professores do Profnit}
\date{2021-10-04}

\usepackage{amsmath,amssymb}
\usepackage{lmodern}
\usepackage{iftex}
\ifPDFTeX
  \usepackage[T1]{fontenc}
  \usepackage[utf8]{inputenc}
  \usepackage{textcomp} % provide euro and other symbols
\else % if luatex or xetex
  \usepackage{unicode-math}
  \defaultfontfeatures{Scale=MatchLowercase}
  \defaultfontfeatures[\rmfamily]{Ligatures=TeX,Scale=1}
\fi
% Use upquote if available, for straight quotes in verbatim environments
\IfFileExists{upquote.sty}{\usepackage{upquote}}{}
\IfFileExists{microtype.sty}{% use microtype if available
  \usepackage[]{microtype}
  \UseMicrotypeSet[protrusion]{basicmath} % disable protrusion for tt fonts
}{}
\makeatletter
\@ifundefined{KOMAClassName}{% if non-KOMA class
  \IfFileExists{parskip.sty}{%
    \usepackage{parskip}
  }{% else
    \setlength{\parindent}{0pt}
    \setlength{\parskip}{6pt plus 2pt minus 1pt}}
}{% if KOMA class
  \KOMAoptions{parskip=half}}
\makeatother
\usepackage{xcolor}
\IfFileExists{xurl.sty}{\usepackage{xurl}}{} % add URL line breaks if available
\IfFileExists{bookmark.sty}{\usepackage{bookmark}}{\usepackage{hyperref}}
\hypersetup{
  pdftitle={Pesquisa Tecnológica: Qualitativa e Quantitativa \& Estatística Aplicada a PI\&TT},
  pdfauthor={Equipe de Professores do Profnit},
  hidelinks,
  pdfcreator={LaTeX via pandoc}}
\urlstyle{same} % disable monospaced font for URLs
\usepackage{longtable,booktabs,array}
\usepackage{calc} % for calculating minipage widths
% Correct order of tables after \paragraph or \subparagraph
\usepackage{etoolbox}
\makeatletter
\patchcmd\longtable{\par}{\if@noskipsec\mbox{}\fi\par}{}{}
\makeatother
% Allow footnotes in longtable head/foot
\IfFileExists{footnotehyper.sty}{\usepackage{footnotehyper}}{\usepackage{footnote}}
\makesavenoteenv{longtable}
\usepackage{graphicx}
\makeatletter
\def\maxwidth{\ifdim\Gin@nat@width>\linewidth\linewidth\else\Gin@nat@width\fi}
\def\maxheight{\ifdim\Gin@nat@height>\textheight\textheight\else\Gin@nat@height\fi}
\makeatother
% Scale images if necessary, so that they will not overflow the page
% margins by default, and it is still possible to overwrite the defaults
% using explicit options in \includegraphics[width, height, ...]{}
\setkeys{Gin}{width=\maxwidth,height=\maxheight,keepaspectratio}
% Set default figure placement to htbp
\makeatletter
\def\fps@figure{htbp}
\makeatother
\setlength{\emergencystretch}{3em} % prevent overfull lines
\providecommand{\tightlist}{%
  \setlength{\itemsep}{0pt}\setlength{\parskip}{0pt}}
\setcounter{secnumdepth}{5}
\usepackage{booktabs}
\ifLuaTeX
  \usepackage{selnolig}  % disable illegal ligatures
\fi
\usepackage[]{natbib}
\bibliographystyle{apalike}

\begin{document}
\maketitle

{
\setcounter{tocdepth}{1}
\tableofcontents
}
\hypertarget{introduuxe7uxe3o}{%
\chapter{Introdução}\label{introduuxe7uxe3o}}

\hypertarget{objetivo-da-proposta}{%
\section{Objetivo da proposta}\label{objetivo-da-proposta}}

Desenvolver material didático complementar para a disciplina \emph{Pesquisa Tecnológica: Qualitativa e Quantitativa \& Estatística Aplicada a PI\&TT}

Pressupostos para elaboração do material didático:

\begin{itemize}
\item
  Material didático consoante às técnicas aplicáveis ao contexto de avaliação de impacto de políticas públicas na área PI \& PP ou correlatas (mas não exclusivo. Trata-se apenas de uma base para a elaboração do material).
\item
  Foco na análise de dados (foco no operacional da pesquisa).
\item
  Desenvolvimento de tutorais para utilização em ferramentas diversas (Excel, SPSS, R e Python), quando aplicável.
\item
  Criação de material didático seguindo o modelo da The WIPO Manual on Open Source Patent Analytics (\url{https://wipo-analytics.github.io/}), além dos slides tradicicionais.
\item
  Os grupos que elaborarem cada aula/tema serão co-autores dos capítulos do ebook.
\end{itemize}

\hypertarget{construuxe7uxe3o-do-material-diduxe1tico-ideias-gerais}{%
\section{Construção do material didático: ideias gerais}\label{construuxe7uxe3o-do-material-diduxe1tico-ideias-gerais}}

\begin{enumerate}
\def\labelenumi{\arabic{enumi}.}
\tightlist
\item
  A proposta é criar, em cada aula (capítulo aqui no livro) uma seção inicial com os conceitos fundamentais e os aspectos formais necessários ao entendimento dos conceitos. A ideia não é replicar os livros especializados, mas fazer uma transposição didática com foco na operacionalização para alunos que nunca viram o tema (partiremos desse pressuposto).
\item
  Na sequência, apresenta-se exemplos de artigos recentes que os conceitos e técnicas foram empregados, mostrando como construir uma análise empregando os conceitos e técnicas de análise.
\item
  Por fim, pode-se construir um tutorial para operacionalização da pesquisa em ferramentas de análise (Excel, SPSS, R, Python e outros). Esta seção poderá ter vários tutoriais para cada tópico em diferentes ferramentas para facilitar a utilização pelos discentes.
\end{enumerate}

\hypertarget{proposta-de-programa-para-organizauxe7uxe3o-das-aulasencontros}{%
\subsection{Proposta de programa para organização das aulas/encontros}\label{proposta-de-programa-para-organizauxe7uxe3o-das-aulasencontros}}

\begin{longtable}[]{@{}
  >{\raggedright\arraybackslash}p{(\columnwidth - 2\tabcolsep) * \real{0.15}}
  >{\raggedright\arraybackslash}p{(\columnwidth - 2\tabcolsep) * \real{0.85}}@{}}
\toprule
Equipe responsável & Aulas \\
\midrule
\endhead
& Aula 1

Métodos e técnicas de coleta de dados em pesquisa qualitativa.

~ \\
& Aula 2

Análise e tratamento de informações em pesquisa qualitativa. Métodos de análise qualitativa e análise de conteúdo.

~ \\
Tadeu & Aula 3

Organização dos dados: coleta, limpeza, tratamento e organização. Bases de dados. \\
& Aula 4

~

Noções de probabilidade. Tipos e natureza das variáveis.

Inferência causal e contrafactuais em avaliação em políticas públicas.

~ \\
Tadeu & Aula 5

Análise exploratória dos dados e Visualização.~ A matriz de variância-covariância. Interpretação dos resultados.~

~

~ \\
& Aula 6

Tipos e técnicas de amostragens.

Seleção aleatória para avaliação de impacto de políticas públicas.

~ \\
& Aula 7

Testes de hipóteses.

Pareamento em avaliação de políticas públicas. \\
& Aula 8

Regressão linear simples e múltipla. PLS: Mínimos Quadrados Parciais. Interpretação dos resultados.

~

~

~ \\
& Aula 9

Modelos de classificação: regressão logística.

~

(Ou Regressão Descontínua, mais utilizada em avaliação de políticas públicas)

~ \\
& Aula 10 -- Diferença em diferenças em avaliação de políticas públicas

~ \\
Tadeu & Aula 11 - Mineração de textos e análise léxica.

~ \\
\bottomrule
\end{longtable}

\hypertarget{muxe9todos-e-tuxe9cnicas-de-coleta-de-dados-em-pesquisa-qualitativa}{%
\chapter{Métodos e técnicas de coleta de dados em pesquisa qualitativa}\label{muxe9todos-e-tuxe9cnicas-de-coleta-de-dados-em-pesquisa-qualitativa}}

\hypertarget{anuxe1lise-e-tratamento-de-informauxe7uxf5es-em-pesquisa-qualitativa.-muxe9todos-de-anuxe1lise-qualitativa-e-anuxe1lise-de-conteuxfado.}{%
\chapter{Análise e tratamento de informações em pesquisa qualitativa. Métodos de análise qualitativa e análise de conteúdo.}\label{anuxe1lise-e-tratamento-de-informauxe7uxf5es-em-pesquisa-qualitativa.-muxe9todos-de-anuxe1lise-qualitativa-e-anuxe1lise-de-conteuxfado.}}

\hypertarget{organizauxe7uxe3o-dos-dados-em-pesquisas-quantitativas-coleta-limpeza-tratamento-e-organizauxe7uxe3o.-bases-de-dados.}{%
\chapter{Organização dos dados em pesquisas quantitativas: coleta, limpeza, tratamento e organização. Bases de dados.}\label{organizauxe7uxe3o-dos-dados-em-pesquisas-quantitativas-coleta-limpeza-tratamento-e-organizauxe7uxe3o.-bases-de-dados.}}

\hypertarget{nouxe7uxf5es-de-probabilidade.-tipos-e-natureza-das-variuxe1veis.-inferuxeancia-causal-e-contrafactuais-em-avaliauxe7uxe3o-em-poluxedticas-puxfablicas.}{%
\chapter{Noções de probabilidade. Tipos e natureza das variáveis. Inferência causal e contrafactuais em avaliação em políticas públicas.}\label{nouxe7uxf5es-de-probabilidade.-tipos-e-natureza-das-variuxe1veis.-inferuxeancia-causal-e-contrafactuais-em-avaliauxe7uxe3o-em-poluxedticas-puxfablicas.}}

\#Análise exploratória dos dados e Visualização.~ A matriz de variância-covariância. Interpretação dos resultados.

\#Tipos e técnicas de amostragens. Seleção aleatória para avaliação de impacto de políticas públicas.

\hypertarget{testes-de-hipuxf3teses.-pareamento-em-avaliauxe7uxe3o-de-poluxedticas-puxfablicas.}{%
\chapter{Testes de hipóteses. Pareamento em avaliação de políticas públicas.}\label{testes-de-hipuxf3teses.-pareamento-em-avaliauxe7uxe3o-de-poluxedticas-puxfablicas.}}

\hypertarget{regressuxe3o-linear-simples-e-muxfaltipla.-pls-muxednimos-quadrados-parciais.-interpretauxe7uxe3o-dos-resultados.}{%
\chapter{Regressão linear simples e múltipla. PLS: Mínimos Quadrados Parciais. Interpretação dos resultados.}\label{regressuxe3o-linear-simples-e-muxfaltipla.-pls-muxednimos-quadrados-parciais.-interpretauxe7uxe3o-dos-resultados.}}

\hypertarget{modelos-de-classificauxe7uxe3o-regressuxe3o-loguxedstica.-ou-regressuxe3o-descontuxednua-mais-utilizada-em-avaliauxe7uxe3o-de-poluxedticas-puxfablicas.}{%
\chapter{Modelos de classificação: regressão logística. Ou Regressão Descontínua, mais utilizada em avaliação de políticas públicas.}\label{modelos-de-classificauxe7uxe3o-regressuxe3o-loguxedstica.-ou-regressuxe3o-descontuxednua-mais-utilizada-em-avaliauxe7uxe3o-de-poluxedticas-puxfablicas.}}

\hypertarget{regressuxe3o-linear-simples-e-muxfaltipla.-pls-muxednimos-quadrados-parciais.-interpretauxe7uxe3o-dos-resultados.-1}{%
\chapter{Regressão linear simples e múltipla. PLS: Mínimos Quadrados Parciais. Interpretação dos resultados.}\label{regressuxe3o-linear-simples-e-muxfaltipla.-pls-muxednimos-quadrados-parciais.-interpretauxe7uxe3o-dos-resultados.-1}}

\hypertarget{minerauxe7uxe3o-de-textos-e-anuxe1lise-luxe9xica.}{%
\chapter{Mineração de textos e análise léxica.}\label{minerauxe7uxe3o-de-textos-e-anuxe1lise-luxe9xica.}}

  \bibliography{book.bib,packages.bib}

\end{document}
